\documentclass{article}
\usepackage{geometry}
\geometry{margin=3cm, vmargin={3cm}}

% useful packages.
\usepackage{amsfonts}
\usepackage{amsmath}
\usepackage{amssymb}
\usepackage{amsthm}
\usepackage{algorithm}
\usepackage{algorithmicx}
\usepackage{algpseudocode}
\usepackage{appendix}
\usepackage{bm}
\usepackage{ctex}
\usepackage{colortbl}
\usepackage{diagbox}
\usepackage{enumerate}
\usepackage{graphicx}
\usepackage{graphicx}       
\usepackage{hyperref}
\usepackage{multicol}
\usepackage{multirow}
\usepackage{indentfirst}
\usepackage{listings}
\usepackage{booktabs}
\usepackage{tabularx}
\usepackage{fancyhdr}
\usepackage{layout}
\usepackage{xcolor}
\usepackage{xeCJK}
\usepackage{subfigure}

% \usepackage{mathpazo} % 设置字体

\usepackage{sysulab}
\setlength{\parindent}{1em}
\renewcommand{\baselinestretch}{1.5}
\newcommand{\ie}{\textit{i.e.}}

% define used colors
\definecolor{GRay}{RGB}{200, 200, 200}

% cancel date demonstration
\date{}

% set experiment report title, used in next.
\newcommand{\rtitle}{实验报告名称}
\newcommand{\lecturename}{课程名称}
\newcommand{\instructorname}{任课教师}
\newcommand{\sgrade}{2018}
\newcommand{\smajor}{计算机科学与技术}
\newcommand{\sname}{你的名字}
\newcommand{\sid}{你的学号}
\newcommand{\semail}{president@mail.sysu.edu.cn} % 填你的邮箱

\begin{document}

%%%% DO NOT MODIFY HERE 
\pagestyle{fancy}
\fancyhead{}
\lhead{\sname}
\chead{\rtitle}
\rhead{\today}

% \maketitle
\centerline{\LARGE \rtitle}
\begin{figure}[h]
    \centering
    \includegraphics[scale=0.7]{sysu_logo.png}
\end{figure} 
% set lecture information
\centerline{Lecture: \textbf{\lecturename} \qquad Instructor: \textbf{\instructorname}}
% set head table
\begin{table}[h]
    \centering
    \begin{tabular}{|c|c|c|c|}
        % \toprule 
        \hline
        \cellcolor{GRay} 年级 & \sgrade & \cellcolor{GRay} 专业 & \smajor \\
        \hline
        \cellcolor{GRay} 学号 & \sid &\cellcolor{GRay} 姓名 & \sname \\
        \hline
        \cellcolor{GRay} 邮箱 & \href{mailto:\semail}{\semail} &\cellcolor{GRay} 日期 & \today \\
        \hline
        % \bottomrule
    \end{tabular}    
\end{table}

%%%% DO NOT MODIFY ABOVE

\section{实验目的}

使用参考文献的示例\cite{vaswani2017attention}.

插入代码的示例:

\begin{lstlisting}[style=sysupython]
def main():
    print("hello world")
\end{lstlisting}
代码高亮风格可在\texttt{sysulab.sty}中修改.
\begin{lstlisting}[style=sysucpp]
#include <iostream>
using namespace std;
int main() {
    printf(123);
}
\end{lstlisting}


%
% References will then be sorted and formatted in the correct style
%
\clearpage

\bibliographystyle{unsrt}
\bibliography{ref}

\end{document}
